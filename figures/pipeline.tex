% System pipeline architecture diagram using TikZ
\begin{tikzpicture}[
    node distance=2cm,
    box/.style={rectangle, draw, fill=blue!20, text width=3cm, text centered, rounded corners, minimum height=1.2cm},
    data/.style={rectangle, draw, fill=green!20, text width=2.5cm, text centered, minimum height=1cm},
    process/.style={rectangle, draw, fill=orange!20, text width=3cm, text centered, rounded corners, minimum height=1.2cm},
    arrow/.style={->, >=stealth, thick}
]

% Input layer
\node[data] (query) {User Query};

% Processing layer
\node[box, below of=query] (embed) {Query Embedding};
\node[process, below of=embed] (retriever) {Document Retriever};
\node[data, below of=retriever] (docs) {Retrieved Documents};

% Generation layer
\node[process, below of=docs] (context) {Context Formation};
\node[box, below of=context] (llm) {Large Language Model};
\node[data, below of=llm] (response) {Generated Response};

% Knowledge base (to the right)
\node[data, right=3cm of retriever] (kb) {Biomedical Knowledge Base};

% Arrows
\draw[arrow] (query) -- (embed);
\draw[arrow] (embed) -- (retriever);
\draw[arrow] (retriever) -- (docs);
\draw[arrow] (docs) -- (context);
\draw[arrow] (context) -- (llm);
\draw[arrow] (llm) -- (response);
\draw[arrow] (kb) -- (retriever);

% Labels on arrows
\node[right, font=\small] at ($(retriever)!0.5!(kb)$) {Search};

\end{tikzpicture}
